%% Generated by Sphinx.
\def\sphinxdocclass{report}
\documentclass[letterpaper,10pt,english]{sphinxmanual}
\ifdefined\pdfpxdimen
   \let\sphinxpxdimen\pdfpxdimen\else\newdimen\sphinxpxdimen
\fi \sphinxpxdimen=.75bp\relax
\ifdefined\pdfimageresolution
    \pdfimageresolution= \numexpr \dimexpr1in\relax/\sphinxpxdimen\relax
\fi
%% let collapsible pdf bookmarks panel have high depth per default
\PassOptionsToPackage{bookmarksdepth=5}{hyperref}

\PassOptionsToPackage{booktabs}{sphinx}
\PassOptionsToPackage{colorrows}{sphinx}

\PassOptionsToPackage{warn}{textcomp}
\usepackage[utf8]{inputenc}
\ifdefined\DeclareUnicodeCharacter
% support both utf8 and utf8x syntaxes
  \ifdefined\DeclareUnicodeCharacterAsOptional
    \def\sphinxDUC#1{\DeclareUnicodeCharacter{"#1}}
  \else
    \let\sphinxDUC\DeclareUnicodeCharacter
  \fi
  \sphinxDUC{00A0}{\nobreakspace}
  \sphinxDUC{2500}{\sphinxunichar{2500}}
  \sphinxDUC{2502}{\sphinxunichar{2502}}
  \sphinxDUC{2514}{\sphinxunichar{2514}}
  \sphinxDUC{251C}{\sphinxunichar{251C}}
  \sphinxDUC{2572}{\textbackslash}
\fi
\usepackage{cmap}
\usepackage[T1]{fontenc}
\usepackage{amsmath,amssymb,amstext}
\usepackage{babel}



\usepackage{tgtermes}
\usepackage{tgheros}
\renewcommand{\ttdefault}{txtt}



\usepackage[Bjarne]{fncychap}
\usepackage{sphinx}

\fvset{fontsize=auto}
\usepackage{geometry}


% Include hyperref last.
\usepackage{hyperref}
% Fix anchor placement for figures with captions.
\usepackage{hypcap}% it must be loaded after hyperref.
% Set up styles of URL: it should be placed after hyperref.
\urlstyle{same}


\usepackage{sphinxmessages}
\setcounter{tocdepth}{1}



\title{Mnemosyne}
\date{Mar 25, 2023}
\release{1}
\author{ntaka19}
\newcommand{\sphinxlogo}{\vbox{}}
\renewcommand{\releasename}{Release}
\makeindex
\begin{document}

\ifdefined\shorthandoff
  \ifnum\catcode`\=\string=\active\shorthandoff{=}\fi
  \ifnum\catcode`\"=\active\shorthandoff{"}\fi
\fi

\pagestyle{empty}
\sphinxmaketitle
\pagestyle{plain}
\sphinxtableofcontents
\pagestyle{normal}
\phantomsection\label{\detokenize{index::doc}}
\noindent{\sphinxincludegraphics[width=80\sphinxpxdimen]{{profile2}.png}\hspace*{\fill}}




\chapter{théma}
\label{\detokenize{index:thema}}
\sphinxstepscope


\section{Interesting Papers on GPT}
\label{\detokenize{src/theme/chatgpt_papers:interesting-papers-on-gpt}}\label{\detokenize{src/theme/chatgpt_papers::doc}}
\sphinxAtStartPar
そもそもChatGPTとLLMとは何かについて説明する。
\begin{enumerate}
\sphinxsetlistlabels{\arabic}{enumi}{enumii}{}{.}%
\item {} 
\sphinxAtStartPar
GPTs are GPTs: An Early Look at the Labor Market Impact Potential of Large Language Models (\sphinxurl{https://arxiv.org/pdf/2303.10130.pdf})
\begin{quote}

\sphinxAtStartPar
Memo:
\begin{itemize}
\item {} 
\sphinxAtStartPar
High wage occupations generally present higher exposure to GPTs.

\item {} \begin{description}
\sphinxlineitem{Some occupations with high exposure:}\begin{itemize}
\item {} 
\sphinxAtStartPar
Interpreters and Translators,

\item {} 
\sphinxAtStartPar
Survey Researchers,

\item {} 
\sphinxAtStartPar
Poets, Lyricists and Creative Writers

\item {} 
\sphinxAtStartPar
Animal Scientists,

\item {} 
\sphinxAtStartPar
Public Relations Specialists,

\end{itemize}

\end{description}

\end{itemize}
\end{quote}

\end{enumerate}

\sphinxstepscope


\section{Interesting Papers on Stable Diffusion}
\label{\detokenize{src/theme/stablediffusion_papers:interesting-papers-on-stable-diffusion}}\label{\detokenize{src/theme/stablediffusion_papers::doc}}\begin{enumerate}
\sphinxsetlistlabels{\arabic}{enumi}{enumii}{}{.}%
\item {} 
\sphinxAtStartPar
High\sphinxhyphen{}resolution image reconstruction with latent diffusion models from human brain activity (\sphinxurl{https://www.biorxiv.org/content/10.1101/2022.11.18.517004v2.full})
\begin{quote}
\begin{itemize}
\item {} 
\sphinxAtStartPar
Reconstructed visual images from MRI scan using diffusion model.

\item {} 
\sphinxAtStartPar
Latent vector representation of text description recovered from fMRI and is feeded to recover image.

\end{itemize}

\sphinxAtStartPar
感想
* PSYCHO\sphinxhyphen{}PASSの世界観に近づいた感じがして面白い。
\end{quote}

\end{enumerate}

\sphinxstepscope


\section{What important truth do very few people agree with you on?}
\label{\detokenize{src/theme/zerotoone:what-important-truth-do-very-few-people-agree-with-you-on}}\label{\detokenize{src/theme/zerotoone::doc}}
\sphinxAtStartPar
考える訓練になるので、ここでいくつかまとめてみる。
まずはZero to Oneのまとめについて。



\sphinxstepscope


\section{théma}
\label{\detokenize{src/theme/index:thema}}\label{\detokenize{src/theme/index::doc}}

\chapter{mathimatiká}
\label{\detokenize{index:mathimatika}}
\sphinxstepscope


\section{Stochastic Calculus}
\label{\detokenize{src/stochasticcalculus/index:stochastic-calculus}}\label{\detokenize{src/stochasticcalculus/index::doc}}
\sphinxstepscope


\subsection{ルベーグ積分}
\label{\detokenize{src/stochasticcalculus/2020-03-16-lebesgue:id1}}\label{\detokenize{src/stochasticcalculus/2020-03-16-lebesgue::doc}}


\sphinxAtStartPar
本ポストは測度論的確率論に関する個人的な定義・定理のまとめ(\sphinxhref{https://www.ms.u-tokyo.ac.jp/~yasuyuki/sem.htm}{ゼミ}ならおそらく暗唱しなければならない事項.ただし厳密性は重視しない.).
主に、吉田洋一先生の「ルベグ積分入門」( %
\begin{footnote}[1]\sphinxAtStartFootnote
\sphinxhref{https://www.amazon.co.jp/dp/B06XGHV4SR/ref=dp-kindle-redirect?\_encoding=UTF8\&btkr=1}{ルベグ積分入門}
%
\end{footnote})を参考にして作成されている.
{[}2{]}\_をもとにしたpdfのテキスト( %
\begin{footnote}[3]\sphinxAtStartFootnote
\sphinxhref{http://www.math.titech.ac.jp/~kawahira/courses/lebesgue.pdf}{ルベーグ積分の基礎のキソ}
%
\end{footnote})があるので、そちらも適宜参考にしたい.

\sphinxAtStartPar
また,途中に確率論の定義を伊藤清先生の確率論をもとに入れる予定 %
\begin{footnote}[4]\sphinxAtStartFootnote
\sphinxhref{https://www.amazon.co.jp/\%E7\%A2\%BA\%E7\%8E\%87\%E8\%AB\%96-\%E5\%B2\%A9\%E6\%B3\%A2\%E5\%9F\%BA\%E7\%A4\%8E\%E6\%95\%B0\%E5\%AD\%A6\%E9\%81\%B8\%E6\%9B\%B8-\%E4\%BC\%8A\%E8\%97\%A4-\%E6\%B8\%85/dp/400007816X}{確率論(岩波書店)}
%
\end{footnote}.
最後に、私自身の解釈も含まれているためことを注意されたい(後にアップデートされる).


\subsubsection{1. 外測度、ルベーグ測度}
\label{\detokenize{src/stochasticcalculus/2020-03-16-lebesgue:id5}}
\sphinxAtStartPar
1.1 外測度 (Outer measure)
\begin{equation*}
\begin{split}m^{\ast}(A)\end{split}
\end{equation*}

\bigskip\hrule\bigskip


\sphinxAtStartPar
まずは一次元の測度を議論する.

\sphinxAtStartPar
外測度は次の5つの条件を満たすように定義する(P. 83),
\begin{equation*}
\begin{split}\begin{equation}
0 \leq m^{\ast}(A) \leq +\infty
\label{eq:11}\tag{C1}
\end{equation}\end{split}
\end{equation*}\begin{equation*}
\begin{split}\begin{equation}
A \subseteq B \text{ならば} m^{\ast}(A) \leqq m^{\ast}(B)
\label{eq:12}\tag{C2}
\end{equation}\end{split}
\end{equation*}\begin{equation*}
\begin{split}\begin{equation}
m^{\ast}\left(\bigcup_{i=1}^{\infty} A_{i}\right) \leq \sum_{i=1}^{\infty} m^{\ast}\left(A_{i}\right)
\label{eq:13}\tag{C3}
\end{equation}\end{split}
\end{equation*}\begin{equation*}
\begin{split}\begin{equation}
m^{\ast}([a,b)) = b-a
\label{eq:14}\tag{C4}
\end{equation}\end{split}
\end{equation*}\begin{equation*}
\begin{split}\begin{equation}
\text{点集合AとBが合同ならば} m^{\ast}(A) = m^{\ast}(B)
\label{eq:15}\tag{C5}
\end{equation}\end{split}
\end{equation*}
\sphinxAtStartPar
注意点として,Eq.
\begin{equation*}
\begin{split}\eqref{eq:13}\end{split}
\end{equation*}
\sphinxAtStartPar
で直和であることを要求しないことがある.
「なるべく広い範囲の点集合」を考えたい.

\sphinxAtStartPar
ここで,外測度を次のように定義すると上記5つの条件が満たせる.

\sphinxAtStartPar
半開区間の列,
\begin{equation*}
\begin{split}\left \{ I_{1},... I_{n},... \right\}\end{split}
\end{equation*}
\sphinxAtStartPar
に対して,
\begin{equation*}
\begin{split}\begin{equation}
m^{\ast}(A):=\inf \left\{ \sum_{n=1}^{\infty} \left|I_{n}\right| : A \subseteq \bigcup_{n=1}^{\infty} I_{n} \right \}
\label{eq:16}\tag{1.1}
\end{equation}\end{split}
\end{equation*}

\paragraph{1.2. 可測集合 (P.96)}
\label{\detokenize{src/stochasticcalculus/2020-03-16-lebesgue:p-96}}
\sphinxAtStartPar
Aを決まった点集合とする.
\begin{equation*}
\begin{split}B \subseteq A\end{split}
\end{equation*}
\sphinxAtStartPar
および
\begin{equation*}
\begin{split}B' \subseteq A^{c}\end{split}
\end{equation*}
\sphinxAtStartPar
であればいつでも,
\begin{equation*}
\begin{split}\begin{equation}
m^{\ast}(B \cup B') = m^{\ast}(B) + m^{\ast}(B')
\label{eq:17}\tag{1.2}
\end{equation}\end{split}
\end{equation*}
\sphinxAtStartPar
が成立するとき,
\begin{equation*}
\begin{split}A\end{split}
\end{equation*}
\sphinxAtStartPar
はルベグ可測であるという.

\sphinxAtStartPar
同値な条件として,
\begin{equation*}
\begin{split}X\end{split}
\end{equation*}
\sphinxAtStartPar
を任意の点集合とする時,
\begin{equation*}
\begin{split}A\end{split}
\end{equation*}
\sphinxAtStartPar
が可測であることは,
\begin{equation*}
\begin{split}\begin{equation}
m^{\ast}(B)=m^{\ast}(B \cap A)+m^{\ast}\left(B \cap A^{c}\right)
\label{eq:18}\tag{1.3}
\end{equation}\end{split}
\end{equation*}
\sphinxAtStartPar
が成立することである. (Eq.
\begin{equation*}
\begin{split}\eqref{eq:18}\end{split}
\end{equation*}
\sphinxAtStartPar
で
\begin{equation*}
\begin{split}B=X\cap A, B'= X\cap A^{c}\end{split}
\end{equation*}
\sphinxAtStartPar
とおく)

\sphinxAtStartPar
可測集合の例を残しておく


\paragraph{1.2.1 可測集合族}
\label{\detokenize{src/stochasticcalculus/2020-03-16-lebesgue:id6}}\begin{equation*}
\begin{split}\begin{equation}
\phi \in \mathcal{M}
\label{eq:121}\tag{M1}
\end{equation}\end{split}
\end{equation*}\begin{equation*}
\begin{split}\begin{equation}
A \in \mathcal{M} \Longrightarrow A^{c} \in \mathcal{M}
\label{eq:122}\tag{M2}
\end{equation}\end{split}
\end{equation*}\begin{equation*}
\begin{split}\begin{equation}
A_{n} \in \mathcal{M} \text{  } (n=1,2,...) \text{ならば,} \bigcup_{i=1}^{\infty} A_{i} \in \mathcal{M}
\label{eq:123}\tag{M3}
\end{equation}\end{split}
\end{equation*}
\sphinxAtStartPar
可測集合に限らない時,一般に\sphinxstylestrong{加法的集合族}と呼ぶ.
\begin{equation*}
\begin{split}\begin{equation}
G\text{が開集合ならば,} G \in \mathcal{M}
\label{eq:124}\tag{M4}
\end{equation}\end{split}
\end{equation*}
\sphinxAtStartPar
さらに可測集合の場合は次が成り立つ.

\sphinxAtStartPar
例.ボレル集合族
\begin{equation*}
\begin{split}\eqref{eq:121}, \eqref{eq:122},\eqref{eq:123},\eqref{eq:124}\end{split}
\end{equation*}
\sphinxAtStartPar
を満たすあらゆる集合を考え,その交わり(=“最小”のもの)をとった集合族
\begin{equation*}
\begin{split}\mathcal{B}\end{split}
\end{equation*}
\sphinxAtStartPar
.


\paragraph{1.3. ルベグ測度}
\label{\detokenize{src/stochasticcalculus/2020-03-16-lebesgue:id7}}\begin{equation*}
\begin{split}A\end{split}
\end{equation*}
\sphinxAtStartPar
が可測であるとき,
\begin{equation*}
\begin{split}\begin{equation}
m(A) = m^{\ast}(A)
\end{equation}\end{split}
\end{equation*}
\sphinxAtStartPar
このとき,
\begin{equation*}
\begin{split}m(A)\end{split}
\end{equation*}
\sphinxAtStartPar
を
\begin{equation*}
\begin{split}A\end{split}
\end{equation*}
\sphinxAtStartPar
のルベグ測度と呼ぶ.
\begin{equation*}
\begin{split}m(A)\end{split}
\end{equation*}
\sphinxAtStartPar
は次の条件を満たす.
\begin{equation*}
\begin{split}\begin{equation}
0 \leq m(A) \leq +\infty
\label{eq:131}\tag{L1}
\end{equation}\end{split}
\end{equation*}\begin{equation*}
\begin{split}\begin{equation}
m\left(\bigcup_{i=1}^{\infty} A_{i}\right) \leq \sum_{i=1}^{\infty} m\left(A_{i}\right)
\label{eq:132}\tag{L2}
\end{equation}\end{split}
\end{equation*}\begin{equation*}
\begin{split}\begin{equation}
m([a,b)) = b-a
\label{eq:133}\tag{L3}
\end{equation}\end{split}
\end{equation*}\begin{equation*}
\begin{split}\begin{equation}
\text{点集合AとBが合同ならば} m(A) = m(B)
\label{eq:134}\tag{L4}
\end{equation}\end{split}
\end{equation*}
\sphinxAtStartPar
外測度が満たすEq.
\begin{equation*}
\begin{split}\eqref{eq:12}\end{split}
\end{equation*}
\sphinxAtStartPar
について記述がない.


\subsubsection{2. 可測関数}
\label{\detokenize{src/stochasticcalculus/2020-03-16-lebesgue:id8}}

\paragraph{2.1. 可測関数と連続関数との関連性}
\label{\detokenize{src/stochasticcalculus/2020-03-16-lebesgue:id9}}\begin{equation*}
\begin{split}f\end{split}
\end{equation*}
\sphinxAtStartPar
が可測集合
\begin{equation*}
\begin{split}A\end{split}
\end{equation*}
\sphinxAtStartPar
を定義域とする関数のとき,どの実数
\begin{equation*}
\begin{split}c\end{split}
\end{equation*}
\sphinxAtStartPar
に対しても,
\begin{equation*}
\begin{split}\begin{equation}
A(f(x) > c) = \{ x | x \in A, f(x) > c \}
\end{equation}\end{split}
\end{equation*}
\sphinxAtStartPar
が可測であるとき**
\begin{equation*}
\begin{split}f\end{split}
\end{equation*}
\sphinxAtStartPar
は
\begin{equation*}
\begin{split}A\end{split}
\end{equation*}
\sphinxAtStartPar
で可測な関数**.


\paragraph{2.2. 確率論の準備}
\label{\detokenize{src/stochasticcalculus/2020-03-16-lebesgue:id10}}
\sphinxAtStartPar
この段階でいくつか確率論の準備ができる. * 確率測度
\begin{itemize}
\item {} 
\sphinxAtStartPar
確率変数の定義
\begin{equation*}
\begin{split}(\Omega,\mathcal{A},P)\end{split}
\end{equation*}
\sphinxAtStartPar
を確率空間として 扱いづらい可測空間から扱いやすい可測空間への写像

\item {} 
\sphinxAtStartPar
確率分布の定義

\end{itemize}


\subsubsection{3. ルベグ積分}
\label{\detokenize{src/stochasticcalculus/2020-03-16-lebesgue:id11}}
\sphinxAtStartPar
正値単関数で定理を各種導出し,それらをもとに,正値関数の定理を導出する(正値関数が導出できれば,一般の関数についても導出可能).
\begin{equation*}
\begin{split}\begin{equation}
A=A_{1} \cup A_{2} \cup \cdots \cup A_{k} (i \neq j \text{ then}, A_{i} \cap A_{j}=\varnothing )
\end{equation}\end{split}
\end{equation*}\begin{equation*}
\begin{split}\begin{equation}
a_{i}=\inf \left\{f(x) | x \in A_{i}\right\} \quad(i=1,2, \cdots, k)
\end{equation}\end{split}
\end{equation*}\begin{equation*}
\begin{split}\begin{equation}
\mathrm{s}=a_{1} m\left(A_{1}\right)+a_{2} m\left(A_{2}\right)+\cdots+a_{k} m\left(A_{k}\right)
\end{equation}\end{split}
\end{equation*}\begin{equation*}
\begin{split}\mathcal{s}\end{split}
\end{equation*}
\sphinxAtStartPar
を
\begin{equation*}
\begin{split}f\end{split}
\end{equation*}
\sphinxAtStartPar
の
\begin{equation*}
\begin{split}A\end{split}
\end{equation*}
\sphinxAtStartPar
における近似和と呼ぶ.
\begin{equation*}
\begin{split}A\end{split}
\end{equation*}
\sphinxAtStartPar
のあらゆる分割
\begin{equation*}
\begin{split}\left\{A_{1}, A_{2}, \cdots, A_{k}\right\}\end{split}
\end{equation*}
\sphinxAtStartPar
について近似和をつくる. これらの集合を
\begin{equation*}
\begin{split}\langle \mathcal{s} \rangle\end{split}
\end{equation*}
\sphinxAtStartPar
と表す.

\sphinxAtStartPar
ここでルベーグ積分の定義は,
\begin{equation*}
\begin{split}\begin{equation}
\int_{A} f(x) d x=\sup \langle \mathcal{s}\rangle
\end{equation}\end{split}
\end{equation*}
\sphinxAtStartPar
である.


\paragraph{3.x 単調収束定理}
\label{\detokenize{src/stochasticcalculus/2020-03-16-lebesgue:x}}

\paragraph{3.2 Fatouの定理 (P. 168)}
\label{\detokenize{src/stochasticcalculus/2020-03-16-lebesgue:fatou-p-168}}\begin{equation*}
\begin{split}\begin{equation}
\int_{A} \displaystyle \varliminf_{n} f_{n} (x) dx \leqq \displaystyle \varliminf_{n} \int_{A} f_{n}(x) dx
\end{equation}\end{split}
\end{equation*}

\paragraph{3.3 Lebesgue の項別積分定理}
\label{\detokenize{src/stochasticcalculus/2020-03-16-lebesgue:lebesgue}}

\paragraph{3.y ルベグ積分とリーマン積分との関係}
\label{\detokenize{src/stochasticcalculus/2020-03-16-lebesgue:y}}
\sphinxAtStartPar
どういった状況で積分可能か?の例


\subsubsection{4. 測度空間、ルベーグ=スティルチェス積分}
\label{\detokenize{src/stochasticcalculus/2020-03-16-lebesgue:id12}}
\sphinxAtStartPar
{\hyperref[\detokenize{7B_7B_20site.baseurl_20_7D_7D/Black-Scholes::doc}]{\sphinxcrossref{Black\sphinxhyphen{}Scholes post}}}


\chapter{michanikí}
\label{\detokenize{index:michaniki}}
\sphinxstepscope

\sphinxAtStartPar
Post(more like a record) on arduino programming.


\section{Arduino}
\label{\detokenize{src/kit/2020-03-13-arduino-network-lamp:arduino}}\label{\detokenize{src/kit/2020-03-13-arduino-network-lamp::doc}}
\sphinxAtStartPar
One of the example which I thought has fruitful information on Aruduino
programming.

\begin{figure}[htbp]
\centering
\capstart

\noindent\sphinxincludegraphics[width=320\sphinxpxdimen,height=480\sphinxpxdimen]{{network_lamp}.mp4}
\caption{ネットワークランプ。テキストの中にある特定のワードの数に応じて表示される色が変わる。文章の中の「感情」を「色」で表すことができる。}\label{\detokenize{src/kit/2020-03-13-arduino-network-lamp:figure}}\end{figure}

\sphinxstepscope


\section{Effective C Sharp \#6.0, \#7.0 まとめ}
\label{\detokenize{src/effectivecsharp/index:effective-c-sharp-6-0-7-0}}\label{\detokenize{src/effectivecsharp/index::doc}}
\sphinxstepscope


\section{Excel Tips}
\label{\detokenize{src/excel/index:excel-tips}}\label{\detokenize{src/excel/index::doc}}
\sphinxstepscope


\section{Software Engineering}
\label{\detokenize{src/softwareengineering/index:software-engineering}}\label{\detokenize{src/softwareengineering/index::doc}}
\sphinxstepscope


\section{Machine Learning System}
\label{\detokenize{src/MLApp/index:machine-learning-system}}\label{\detokenize{src/MLApp/index::doc}}

\subsection{StreamLit}
\label{\detokenize{src/MLApp/index:streamlit}}

\subsection{Docker}
\label{\detokenize{src/MLApp/index:docker}}

\chapter{oikonomía}
\label{\detokenize{index:oikonomia}}
\sphinxstepscope


\section{Securities Trade Lifecycle}
\label{\detokenize{src/securitiestradelifecycle/index:securities-trade-lifecycle}}\label{\detokenize{src/securitiestradelifecycle/index::doc}}\begin{enumerate}
\sphinxsetlistlabels{\arabic}{enumi}{enumii}{}{.}%
\item {} 
\sphinxAtStartPar
Trade execution

\item {} 
\sphinxAtStartPar
Trade capture

\item {} 
\sphinxAtStartPar
Trade enrichment

\item {} 
\sphinxAtStartPar
Trade confirmation

\end{enumerate}
\begin{itemize}
\item {} 
\sphinxAtStartPar
SWIFT:

\end{itemize}

\sphinxstepscope


\section{Deep Hedge}
\label{\detokenize{src/deephedge/index:deep-hedge}}\label{\detokenize{src/deephedge/index::doc}}
\sphinxAtStartPar
Deep Hedgeについてまとめる。
まず通常のヘッジでは、


\subsection{Deep Hedge}
\label{\detokenize{src/deephedge/index:id1}}
\sphinxAtStartPar
This link might be useful.
\sphinxurl{https://github.com/YuMan-Tam/deep-hedging}


\subsection{PfHedge}
\label{\detokenize{src/deephedge/index:pfhedge}}

\chapter{nevroepistími}
\label{\detokenize{index:nevroepistimi}}
\sphinxstepscope


\section{Neuroscience}
\label{\detokenize{src/neuroscience/index:neuroscience}}\label{\detokenize{src/neuroscience/index::doc}}
\sphinxstepscope


\subsection{Hebbian Learning}
\label{\detokenize{src/neuroscience/hebb:hebbian-learning}}\label{\detokenize{src/neuroscience/hebb::doc}}
\sphinxAtStartPar
強化学習とヘブ学習の関連性についていくつか述べる。

\sphinxAtStartPar
ここに私が以前書いたレビューがある.

\phantomsection\label{\detokenize{src/neuroscience/hebb:pdf-link}}
\sphinxAtStartPar
Here’s my \sphinxhref{../../\_static/RLHebb.pdf}{link}.


\chapter{Tips}
\label{\detokenize{index:tips}}
\sphinxstepscope


\section{Tips}
\label{\detokenize{src/Tips/index:tips}}\label{\detokenize{src/Tips/index::doc}}
\sphinxAtStartPar
\sphinxurl{https://www.sphinx-doc.org/ja/master/usage/restructuredtext/directives.html}
\sphinxSetupCaptionForVerbatim{Markdown to Restructured Text (RST)}
\def\sphinxLiteralBlockLabel{\label{\detokenize{src/Tips/index:id1}}}
\begin{sphinxVerbatim}[commandchars=\\\{\},numbers=left,firstnumber=1,stepnumber=1]
pandoc\PYG{+w}{ }\PYGZhy{}f\PYG{+w}{ }markdown\PYG{+w}{ }\PYGZhy{}t\PYG{+w}{ }rst\PYG{+w}{ }\PYGZhy{}o\PYG{+w}{ }README.rst\PYG{+w}{ }README.md
\end{sphinxVerbatim}


\chapter{Indices and tables}
\label{\detokenize{index:indices-and-tables}}\begin{itemize}
\item {} 
\sphinxAtStartPar
\DUrole{xref,std,std-ref}{genindex}

\item {} 
\sphinxAtStartPar
\DUrole{xref,std,std-ref}{modindex}

\item {} 
\sphinxAtStartPar
\DUrole{xref,std,std-ref}{search}

\end{itemize}



\renewcommand{\indexname}{Index}
\printindex
\end{document}